\documentclass{article}
\usepackage{braket,amsmath,graphicx,caption}
\usepackage[margin=0.7in]{geometry}

\begin{document}
\title{\bf Lightning-quick introduction to dynamical mean field theory}
\author{Abhirup Mukherjee}
\maketitle
\abstract{This is a very short introduction to the philosophy and algorithm of dynamical mean field theory (DMFT). I brought these points together and wrote this up mostly to cement my own understanding of the topic.}

\section{Refresher on (static) mean field theory}
The Curie-Weiss version of mean field theory involves replacing the spatial fluctuations in the Hamiltonian or the energy by an effective static field. The static field has to be determined self-consistently. To see what this means, we take the canonical example of the Ising model. Its Hamiltonian is given by
\begin{equation}\begin{aligned}
	H = J\sum_{\left<ij \right>} S_i^z S_j^z = J\sum_i S_i^z \sum_{j \in \text{NN of }i}S_j^z
\end{aligned}\end{equation}
In order to introduce the mean-field, we replace the spins \(S_j^z\) of the nearest-neighbour sites by their average value \(\left<S_j^z\right> \equiv m_j\):
\begin{equation}\begin{aligned}
	H_\text{MF} = J\sum_i S_i^z \sum_{j \in \text{NN of }i}m_j
\end{aligned}\end{equation}
Because of translation symmetry, we expect the average local magnetisation to be independent of the position \(j\): \(m_j \equiv m_\text{loc}\). If \(z\) is the coordination number of the lattice, we get
\begin{equation}\begin{aligned}
	H_\text{MF} = J\sum_i S_i^z z m = h_\text{MF} \sum_i S_i^z,
\end{aligned}\end{equation}
where we have defined the static mean field \(h_\text{MF} \equiv Jzm_\text{loc}\). This mean field Hamiltonian is solvable, in terms of \(h_\text{MF}\). The mean-field itself, however, is still unknown. To determine it, we will use the fact that if our approach is to be internally consistent, the average magnetisation \(\left<S_i^z \right>\) obtained from the mean-field Hamiltonian \(H_\text{MF}\) should be equal to that defined before, \(m_\text{loc}\). This is again demanded on grounds of translation invariance. Since \(H_\text{MF}\) just consists of decoupled spins, the local Hamiltonian has two solutions: \(S_i^z = \pm \frac{1}{2}\) with energies \(\pm \frac{h_\text{MF}}{2}\). The local partition function \(Z_\text{MF}\) and hence the magnetisation at site \(i\) is then obtained easily:
\begin{equation}\begin{aligned}
	Z_\text{MF} = 2\cosh \left(\beta h_\text{MF}/2\right), m = \frac{1}{Z_\text{MF}}\sum_{S_i^z = \pm \frac{1}{2}} S_i^z e^{-\beta h_\text{MF}S_i^z} = \tanh \left(\beta h_\text{MF}/2\right)
\end{aligned}\end{equation}
The self-consistency equation takes the form
\begin{equation}\begin{aligned}
	m_\text{loc} = m = \tanh \left(\beta Jzm_\text{loc}/2\right)
\end{aligned}\end{equation}
This equation now has to be solved numerically, to obtain the value of the local magnetisation \(m_\text{loc}\). This is the general approach to the mean field approximation: 
\begin{itemize}
	\item replace the fluctuations by an effective local mean field \(y_\text{loc}\), in order to obtain a simpler problem (the simplified Hamiltonian \(H_\text{MF}\),
	\item solve the simpler problem at a particular local site \(i\) to calculate the mean field \(y_i\) from it, and
	\item demand that a self-consistent solution should have \(y_i = y_\text{loc}\), and solve this equation.
\end{itemize}



\end{document}
